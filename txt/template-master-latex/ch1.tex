
%\setcounter{page}{\value{page}}
\chapter{Introducere}
\label{cap:Introducere}
%\setcounter{page}{7}
%\addtocounter{page}{6}
\pagestyle{headings}
\section{Introducere}
\par \textbf{Acest capitol definește în secțiunea 1.2 problema adresată, fiind subliniați principalii piloni ai problemei: costul serviciilor educaționale și în special complexitatea și costul ridicat al soluțiilor existente ce implică tehnologia mediilor de învățare 3D. Secțiunea 1.3 este un enunț al soluției ce se dorește a fi oferită în această lucrare. Secțiunea 1.4 subliniază etaplele studiului iar secțiunea 1.5 tratează în linii mari teorii referitoare la mediile de învățare 3D și la învățare în general, teorii ce vor avea influență asupra produsului final (applicația informatică).}
\section{Problema adresată}
\par Asigurarea unui învățămînt de calitate poate implica costuri semnificative pentru diversele instituții sau organizații care oferă asemenea servicii, în special când mediul optim de învățare presupune colaborarea, întreprinderea de experimente multiple sau lucrul în medii cu risc ridicat. De asemenea, în unele cazuri, instruirea poate avea loc în medii simulate al căror cost de construire și folosire ar fi mult prea mare.
\section{Motivație}
\par Ca răspuns la complexitatea și costul soluțiilor existente, se dorește realizarea unei aplicații reprezentând o un mediu virtual 3D; o aplicație de dimensiuni reduse dar care poate fi cu ușurință extinsă și completată cu noi elemente reprezentând: experimente de laborator virtual, diverse forme de reprezentare în forma grafica 3D a cunoștințelor, spații virtuale pentru desfașurarea de activități educative (muzee, săli de conferință virtuală) etc. 
O asemenea aplicație va avea se va dezvolta în două module: client și server, ambele dinamice, ușor de extins și care să se constituie o mini platformă pe care alți programatori să poată construi cu minim efort lumi virtuale orientate spre livrarea conținutului educativ intr-o formă cât mai atragatoare pentru utilizatori.
\par Aplicația informatică va rula sub GNU/Linux, pentru inplementarea acesteia se vor folosi doar unelte și bibilioteci software dezvoltate de comunitățile free software / open source. Codul sursa va fi eliberat cu o licență liberă.
\section{Descrierea studiului pe capitole}
\par În \textbf{\textit{Capitolul 1}} (capitolul curent) se pune accentul pe descrierea nevoii instituțiilor furnizoare de servicii educațional de unelte moderne pentru îndeplinirea obiectivelor lor specifice și pe lipsa de alternative care să indeplinească  \textbf{simultan} următoarele aspecte: libertatea codului (open source / free software), dimensiune redusă și modularitate ridicată, costuri minime. Se afirmă obiectivul creării unui program liber (free/open source) ușor de menținut și extins reprezentând o lume virtuală 3D. Definițiile și conceptele teoriei învățării in medii virtuale încheie capitolul 1.
\par În \textbf{\textit{Capitolul 2}} sunt enumerate obiectivele cercetării.
\par În \textbf{\textit{Capitolul 3}} sunt prezentate specificațiile generale ale lucrării de cercetare, atât obiectivele minimale care vor trebui atinse până la definitivarea studiului cât și obiectivele potențiale de atins în cazul extinderii lucrării.
\par În \textbf{\textit{Capitolul 4}} se analizează sistemul în întregime prin descrierea ansamblului \textit{client-server}. Se decide design-ul  sistemului și se argumentează alegerea făcută prin testele anterior stabilite.
\par În \textbf{\textit{Capitolul 5}} se realizează un studiu amănunțit al aplicației \textit{server} din sistem. Se decide design-ul aplicației, se efectuează teste și se decid aspectele tehnice referitoare la implementarea aplicației: limbajul folosit pentru implementare, librării software folosite, algoritmi și metode de implementare.
\par În \textbf{\textit{Capitolul 6}} se realizează un studiu amănunțit al aplicației \textit{client} din sistem. Se decide design-ul aplicației, se efectuează teste și se decid aspectele tehnice referitoare la implementarea aplicației: limbajul folosit pentru implementare, librării software folosite, algoritmi și metode de implementare.
\par În \textbf{\textit{Capitolul 7}} sunt prezentate concluziile.
\section{Învațarea și mediile 3D. Concepte relevante.}
\par În mare masură, soluția tuturor acestor probleme se găsește în mediile de învățare virtuale cu redare 3D, datorită unor caracterisici care le califică ca și cadru (în unele cazuri ideal) de învățare. Unele dintre cele mai importante caracteristici ale mediilor de învățare cu redare tridimensionala sunt: capacitatea de a simula orice spațiu fizic, de a intermedia interacțiunea dintre diverse persoane aflate în zone geografice diferite și oferta de unelte de observare și măsurare a performanțelor sau a progresului participanților la procesul de învățare și faptul ca orice resursă virtuală poate fi refolosită fară costuri suplimentare.
\par O analiză a autorilor Wann și Mon Williams oferă o descriere a mediilor tridimensionale ca medii ce ”valorifică aspectele naturale ale percepției umane prin extinderea informațiilor vizuale în trei dimensiuni spațiale și care poate suplimenta aceasta informație cu alți stimuli și modificari temporale”\cite{C01} și care ”permit interacțiunea utilizatorului cu obiectele redate”\cite{C01}. Se pot astfel deduce trei elemente care disting mediile de învățare 3D de alte medii de învățare virtuale. 	Mai detaliat, mediile virtuale tridimensionale sunt medii grafice ce crează impresia de spațiu 3D, în care utilizatorul controlează caractere generate de computer (avatare), caractere care îi reprezintă în timp ce interacționează cu mediul sau cu alți utilizatori. Acestea pot contribui la sistemul educațional prin facilitarea colaborării, comunicării și experimentării. Prin oferirea unui surogat al realității, mediile de învățare pot crea percepția de existență a utilizatorului în mediul simulat. 
\par În psihologie și educație, învățarea este definită ca fiind procesul care aduce împreună experiența cognitivă, emoțională și influența de mediu, pentru acumularea, îmbunatațirea sau schimbarea cunoștințelor, abilităților sau a concepției despre lume a unui individ.[C02]
\par Clasificarea metodelor de învățare, stabilită de Frederic Vester[c03] : învățarea auditivă, învățarea vizuală, învățarea tactilă, învățarea cognitivă ( prin intelect ). Prin această metodă de clasificare F. Vester, neagă efortul intelectual pentru primele trei tipuri, acest efort fiind atribuit învățării cognitive. Deși nu poate fi în totalitate adevărat, fiecare dintre noi am putut experimenta reducerea 'consumului' intelectual atunci când am învățat folosindu-ne de materiale didactice cu vizuale sau auditive. Personele de toate vârstele învață cel mai bine atunci cînd sunt implicate în experiențe semnificative. Învățarea are loc atunci când mintea este capabilă să pună la un loc informațiile primite de la toate simțurile și să le coreleze cu experințele trecute. Prin folosirea mai multor simțuri pentru a învăța se poate da mai mult sens procesului de acumulare. Copii în mod natural învață folosindu-se de toate simțurile în cel mai eficient mod posibil.
\par Valoarea unui mediu 3D bine construit constă în faptul ca poate antrena simțul vizual al utilizatorului, cu îmbunătățirea rezultatelor. 